% Created 2025-05-21 Wed 22:41
% Intended LaTeX compiler: pdflatex
\documentclass[11pt]{article}
\usepackage[utf8]{inputenc}
\usepackage[T1]{fontenc}
\usepackage{graphicx}
\usepackage{longtable}
\usepackage{wrapfig}
\usepackage{rotating}
\usepackage[normalem]{ulem}
\usepackage{amsmath}
\usepackage{amssymb}
\usepackage{capt-of}
\usepackage{hyperref}
\usepackage{mathptmx}  % Times font
\usepackage{helvet}   % Helvetica font
\renewcommand{\familydefault}{\sfdefault} % Sans-serif as default
\usepackage{titlesec}
\usepackage{lmodern}
\usepackage{amsmath}
\author{Serob Tigranyan}
\date{\today}
\title{Linear Algebra}
\hypersetup{
 pdfauthor={Serob Tigranyan},
 pdftitle={Linear Algebra},
 pdfkeywords={},
 pdfsubject={},
 pdfcreator={Emacs 30.1 (Org mode 9.7.26)}, 
 pdflang={English}}
\begin{document}

\maketitle
\tableofcontents

\newpage
\section{Task 1}
\label{sec:org8a184da}
Given the following basis vectors:
\[
e_1=(1,1,3)
\]
\[
e_2=(1,1,0)
\]
\[
e_3=(1,2,0)
\]

And vector \(x\):
\[
x=(5,4,6)
\]
\subsection{Linear Independence}
\label{sec:org41480b3}
We need to check if the given basis vectors are valid:
\[
A =
\begin{bmatrix}
1 & 1 & 1 \\
1 & 1 & 2 \\
3 & 0 & 0
\end{bmatrix}
\]

Determine the Rank by first finding the determinant:
\[
D(A) = 0 + 6 + 0 -3 - 0 -0 = 3
\]

Since our determinant is non-zero, then we have a full rank which is 3 which further means that the basis vectors are linearly independent. 
\[
rank(A) = 3
\]
\subsection{Vector expressed in given basis}
\label{sec:org1d42f87}
We need to express vector \(x\) in our basis \(e_1\), \(e_2\), \(e_3\).
\[
x=\lambda_1 e_1 + \lambda_2 e_2 + \lambda_3 e_3
\]

Therefore we have:
\[
(5,4,6)=\lambda (1,1,3) + \lambda (1,1,0) + \lambda (1,2,0)
\]
\[
(5,4,6)=(\lambda_1 + \lambda_2 + \lambda_3; \lambda_1 + \lambda_2 + 2 \lambda_3; 3 \lambda_1)
\]

We form a linear system of equations:
\[
\left\{
\begin{aligned}
\lambda_1 + \lambda_2 + \lambda_3 = 5 \\
\lambda_1 + \lambda_2 + 2 \lambda_3 = 4 \\
3 \lambda_1 = 6
\end{aligned}
\]

Now we solve the system:
\[
\left\{
\begin{aligned}
\lambda_1 + \lambda_2 + \lambda_3 = 5 \\
\lambda_1 + \lambda_2 + 2 \lambda_3 = 4 \\
\lambda_1 = 2
\end{aligned}
\]
\[
\left\{
\begin{aligned}
\lambda_2 = 3 - \lambda_3 \\
2 + (3 - \lambda_3) + 2 \lambda_3 = 4 \\
\lambda_1 = 2
\end{aligned}
\]
\[
\left\{
\begin{aligned}
\lambda_2 = 3 - \lambda_3 \\
\lambda_3 = -1 \\
\lambda_1 = 2
\end{aligned}
\]
\[
\left\{
\begin{aligned}
\lambda_2 = 4 \\
\lambda_3 = -1 \\
\lambda_1 = 2
\end{aligned}
\]

Therefore our vector \(x\) expressed in our basis vectors \(e_1\), \(e_2\), \(e_3\) is:
\[
x_e = (2,4,-1)
\]

\newpage
\section{Task 2}
\label{sec:org3879951}
Given the vectors \(a_1\) and \(a_2\) we need to find if they're orthogonal and if so, then find their orthogonal basis vectors:
\[
a_1 = (4,4,1,-1)
\]
\[
a_2 = (1,-1,1,1)
\]
\subsection{Orthogonality}
\label{sec:orgafa7d12}
To check if these two are orthogonal we perform scalar multiplication on both of them:
\[
(a_1 \cdot a_2) = 4 - 4 + 1 - 1 = 0 \Rightarrow a_1 \perp a_2
\]

Our vectors are indeed orthogonal, now we need to find their orthogonal basis vectors.
\subsubsection{Orthogonal Basis Vectors}
\label{sec:org3b7daa2}
To find the orthogonal basis vectors we pick arbitrary vector:
\[
\gamma=(x,y,z,w)
\]

So that:
\[
\gamma \perp a_1
\]
\[
\gamma \perp a_2
\]

We form a linear system of equations from this like so:
\[
\left\{
\begin{aligned}
(\gamma \cdot a_1) = 0 \\
(\gamma \cdot a_2) = 0
\end{aligned}
\Rightarrow
\left\{
\begin{aligned}
4x+4y+z-w = 0 \\
x-y+z+w = 0
\end{aligned}
\]

Now we try and solve the system:
\[
\left\{
\begin{aligned}
4x+4y+z-w = 0 \\
x-y+z+w = 0
\end{aligned}
\]
\]

To solve the system we'll need to find the rank:
\[
A =
\begin{bmatrix}
4 & 4 & 1 & -1 \\
1 & -1 & 1 & 1
\end{bmatrix}
\]

Find the deltas of matrix $A$:
\[
\Delta_1 = 4 \neq 0
\]
\[
\Delta_2 =
\begin{bmatrix}
4 & 4 \\
1 & -1
\end{bmatrix}
= -4 - 4 = -8 \neq 0
\Rightarrow
rank(A) = 2
\]

Now we need the identity matrix based on our rank:
\[
I =
\begin{bmatrix}
1 & 0 \\
0 & 1
\end{bmatrix}
\]

Now we solve our system based on the identity matrix. \\
Because our identity matrix is 2x2, we move two elements from the left to right:
\[
\left\{
\begin{aligned}
4x+4y = -z+w \\
x-y = -z-w
\end{aligned}
\]

Now solve by placing the first row elements to be on the right side of the equations:
\[
\left\{
\begin{aligned}
4x+4y = -1 \\ 
x-y = -1
\end{aligned}
\]
\[
\left\{
\begin{aligned}
4x+4y = -1 \\ 
x = -1 + y
\end{aligned}
\]
\[
\left\{
\begin{aligned}
y = \frac{3}{8} \\ 
x = -1 + y
\end{aligned}
\]
\[
\left\{
\begin{aligned}
y = \frac{3}{8} \\ 
x = -\frac{5}{8}
\end{aligned}
\]

Now the second row of elements to be on the right side of the equations:
\[
\left\{
\begin{aligned}
4x+4y = 1 \\ 
x-y = -1
\end{aligned}
\Rightarrow
\left\{
\begin{aligned}
y = \frac{5}{8} \\
x = -\frac{3}{8} \\
\end{aligned}
\]

Now we have to basis vectors $\gamma_1$ and $\gamma_2$:
\[
\gamma_1 = \left( -\frac{5}{8}; \frac{3}{8}; 1; 0 \right)
\]
\[
\gamma_2 = \left( -\frac{3}{8}; \frac{5}{8}; 0; 1 \right)
\]

Now we check if they're orthogonal: 
\[
(\gamma_1 \cdot \gamma_2) = \left( \frac{5}{8} \cdot \frac{3}{8} + \frac{3}{8} \cdot \frac{5}{8} + 0 + 0 \right) = \frac{15}{32} \neq 0
\]

Unfortunately our vectors are not orthogonal therefore we must find the fourth $\beta$. \\
We do this by picking any of the previous vectors to serve as $a_3$, in this case $a_3 = \gamma_1$:
\[
\beta = (x,y,z,w) \neq 0
\]
\[
\beta \perp a_1, \beta \perp a_2, \beta \perp a_3
\]

Form a linear system of equations:
\[
\left\{
\begin{aligned}
(\beta \cdot a_1) = 0 \\
(\beta \cdot a_2) = 0 \\
(\beta \cdot a_3) = 0
\end{aligned}
\Rightarrow
\left\{
\begin{aligned}
4x + 4y + z - w = 0 \\
x - y + z + w = 0 \\
-\frac{5x}{8}+ \frac{3y}{8}+z = 0
\end{aligned}
\]

To solve we'll need to turn this into a matrix and find the rank again:
\[
A =
\begin{bmatrix}
4 & 4 & 1 & -1 \\
1 & -1 & 1 & 1 \\
-\frac{5}{8} & \frac{3}{8} & 1 & 0
\end{bmatrix}
\]

Find the rank:
\[
\Delta_3 = 
\begin{bmatrix}
4 & 4 & 1 \\
1 & -1 & 1 \\
-\frac{5}{8} & \frac{3}{8} & 1
\end{bmatrix}
= -\frac{49}{4} \neq 0
\Rightarrow
rank(A) =  3
\]

Now form a linear system of equations again:
\[
\left\{
\begin{aligned}
4x + 4y + z = w \\
x-y+z=-w \\
-\frac{5}{8}+\frac{3y}{8}+z=0
\end{aligned}
\]

We can solve this system of equations using Cramer's Rule:
\[
\begin{bmatrix}
4 & 4 & 1 \\
1 & -1 & 1 \\
-\frac{5}{8} & \frac{3}{8} & 1
\end{bmatrix}
\cdot
\begin{bmatrix}
x \\ y \\ z 
\end{bmatrix}
\]
\[
D(A) = -\frac{49}{4}
\]
\[
x = D \left(
\begin{bmatrix}
1 & 4 & 1 \\
-1 & -1 & 1 \\
0 & \frac{3}{8} & 1
\end{bmatrix}
\right) = \frac{9}{4}
\]
\[
y = D \left(
\begin{bmatrix}
4 & 1 & 1 \\
1 & -1 & 1 \\
-\frac{5}{8} & 0 & 1
\end{bmatrix}
\right) = -\frac{25}{4}
\]
\[
z = D \left(
\begin{bmatrix}
4 & 4 & 1 \\
1 & -1 & -1 \\
-\frac{5}{8} & \frac{3}{8} & 0
\end{bmatrix}
\right) = \frac{15}{4}
\]

Therefore we get our vector $\beta$ which is our fourth basis vector $a_4$:
\[
a_4 = \beta = \left( -\frac{9}{49} ; \frac{25}{49} ; -\frac{15}{49} ; 1 \right)
\]

We conclude with our orthogonal basis vectors:
\[
a_1 = (4;4;1;-1)
\]
\[
a_2 = (1;-1;1;1)
\]
\[
a_3 = (-\frac{5}{8};\frac{3}{8};1;0)
\]
\[
a_4 = (-\frac{9}{49};\frac{25}{49};-\frac{15}{49};1) 
\]
\end{document}
