% Created 2025-05-18 Sun 20:12
% Intended LaTeX compiler: pdflatex
\documentclass[11pt]{article}
\usepackage[utf8]{inputenc}
\usepackage[T1]{fontenc}
\usepackage{graphicx}
\usepackage{longtable}
\usepackage{wrapfig}
\usepackage{rotating}
\usepackage[normalem]{ulem}
\usepackage{amsmath}
\usepackage{amssymb}
\usepackage{capt-of}
\usepackage{hyperref}
\usepackage{mathptmx}  % Times font
\usepackage{helvet}   % Helvetica font
\renewcommand{\familydefault}{\sfdefault} % Sans-serif as default
\usepackage{titlesec}
\usepackage{lmodern}
\usepackage{amsmath}
\author{Serob Tigranyan}
\date{\today}
\title{Linear Algebra}
\hypersetup{
 pdfauthor={Serob Tigranyan},
 pdftitle={Linear Algebra},
 pdfkeywords={},
 pdfsubject={},
 pdfcreator={Emacs 29.4 (Org mode 9.7.11)}, 
 pdflang={English}}
\begin{document}

\maketitle
\tableofcontents

\newpage
\section{Task 1}
\label{sec:org11759b6}
Given the following basis vectors:
\[
e_1=(1,1,1)
\]
\[
e_2=(1,1,2)
\]
\[
e_3=(1,-1,1)
\]

And vector x:
\[
x=(6,9,14)
\]

Check if the basis vectors are linearly independent by placing each vector as a column in matrix and calculating the \textbf{rank}:
\[
 A =
 \begin{bmatrix}
 1 & 1 & 1 \\
 1 & 1 & -1 \\
 1 & 2 & 1
 \end{bmatrix}
\]

Calculate rank:
\[
\Delta_1 = 1 \cdot 1 \cdot 1 + 1 \cdot (-1) \cdot 1 + 1 \cdot 1 \cdot 2 - 1 \cdot 1 \cdot 1 - 1 \cdot 1 \cdot 1 - 1 \cdot (-1) \cdot 2 = 2
\]
\[
rank(A) = 3
\]

Because the \textbf{rank} is 3, the same amount as the length of the vectors, this means that the basis vectors are linearly independent.

\(x\) expressed as a vector inside our basis:
\[
x = c_1 e_1 + c_2 e_2 + c_3 e_3
\]

\[
 \left\{
 \begin{align}
 c_1 + c_2 + c_3 = x_1 \\
 c_1 + c_2 - c_3 = x_2 \\
 c_1 + 2c_2 + c_3 = x_3
 \end{align}
\]
\end{document}
